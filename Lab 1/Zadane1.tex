\documentclass[11pt]{article}
\usepackage{amsmath,amssymb,amsthm}
\usepackage{algorithm}
\usepackage[noend]{algpseudocode} 

%---enable russian----

\usepackage[utf8]{inputenc}
\usepackage[russian]{babel}

\usepackage{hyperref} % url env

% PROBABILITY SYMBOLS
\newcommand*\PROB\Pr 
\DeclareMathOperator*{\EXPECT}{\mathbb{E}}


% Sets, Rngs, ets 
\newcommand{\N}{{{\mathbb N}}}
\newcommand{\Z}{{{\mathbb Z}}}
\newcommand{\R}{{{\mathbb R}}}
\newcommand{\Zp}{\ints_p} % Integers modulo p
\newcommand{\Zq}{\ints_q} % Integers modulo q
\newcommand{\Zn}{\ints_N} % Integers modulo N

% Landau 
\newcommand{\bigO}{\mathcal{O}}
\newcommand*{\OLandau}{\bigO}
\newcommand*{\WLandau}{\Omega}
\newcommand*{\xOLandau}{\widetilde{\OLandau}}
\newcommand*{\xWLandau}{\widetilde{\WLandau}}
\newcommand*{\TLandau}{\Theta}
\newcommand*{\xTLandau}{\widetilde{\TLandau}}
\newcommand{\smallo}{o} %technically, an omicron
\newcommand{\softO}{\widetilde{\bigO}}
\newcommand{\wLandau}{\omega}
\newcommand{\negl}{\mathrm{negl}} 

% Misc
\newcommand{\eps}{\varepsilon}
\newcommand{\inprod}[1]{\left\langle #1 \right\rangle}



\newcommand{\handout}[5]{
	\noindent
	\begin{center}
		\framebox{
			\vbox{
				\hbox to 5.78in { {\bf Научно-исследовательская практика} \hfill #2 }
				\vspace{4mm}
				\hbox to 5.78in { {\Large \hfill #5  \hfill} }
				\vspace{2mm}
				\hbox to 5.78in { {\em #3 \hfill #4} }
			}
		}
	\end{center}
	\vspace*{4mm}
}

\newcommand{\lecture}[4]{\handout{#1}{#2}{#3}{Scribe: #4}{Теоретикочисловые функции}}

\newtheorem{theorem}{Теорема}
\newtheorem{lemma}{Лемма}
\newtheorem{definition}{Определение}
\newtheorem{corollary}{Следствие}
\newtheorem{fact}{Факт}

% 1-inch margins
\topmargin 0pt
\advance \topmargin by -\headheight
\advance \topmargin by -\headsep
\textheight 8.9in
\oddsidemargin 0pt
\evensidemargin \oddsidemargin
\marginparwidth 0.5in
\textwidth 6.5in

\parindent 0.1in
\parskip 1.5ex

\begin{document}
	
	\lecture{}{Лето 2020}{}{Сацута Анатолий}
	
	\section*{Функции $\tau$ и $\sigma$}
	
	Сумма данных целых чисел
	\[
		\sigma(180)=\frac{2^3-1}{2-1}\frac{3^3-1}{3-1}\frac{5^2-1}{5-1}=\frac71\frac{26}2\frac{24}{4}=7\cdot13\cdot6.
	\]
	Одно из наиболее интересных свойств функции делителя $\tau$ это то, что произведение положительных делителей целого числа $n>1$ равно $n^{\tau(n)/2}$. Это не так сложно доказать: обозначим символом $d$ произвольный положительный делитель числа $n$, пусть $n=dd'$ для какого-нибудь $d'$. Так $d$ находится в диапазоне $\tau(d)$ положительных делителей числа $n$. Уравнения $\tau(d)$ встречаются. Перемножив их, получим: 
	\[
		n^{\tau(n)}=\prod_{d|n}^{}d\cdot\prod_{d'|n}^{}d'
	\]
	Так как $d$ является делителем $n$, то и $d'$ делитель $n$; следовательно, $\prod_{d|n}^{}d=\prod_{d'|n}^{}d'$. Таким образом, сейчас у нас такая ситуация:
	\[
		n^{\tau(n)}=\left(\prod_{d|n}^{}d\right)^2
	\]
	что равняется:
	\[
		n^{\tau(n)/2}=\prod_{d|n}^{}d
	\]	
	Читатель может (или,во всяком случае, должен) иметь сомнения относительно этого уравнения, потому что отнюдь не очивидно, что левая часть равенства всегда целое число. Если $\tau(n)$ чётно, то проблем, очевидно, мы не имеем. Когда же $\tau(n)$ нечётно, $n$ оказывется идеальным квадратом (проблема 7), предположим, что $n=m^2$ , таким образом, $n^{\tau(n)/2}=m^{\tau(n)}$ --- устраняет все подозрения.
	Приведём числовой пример, произведение пяти делителей числа 16(т.е. 1,2,4,8,16) это
	\[
		\prod_{d|16}^{}d=16^{\tau(16)/2}=16^{5/2}=4^5=1024.
	\]
	Мультипликативные функции естественным образом возникают при изучении простой факторизации целого числа. Прежде чем дать определение, заметим, что
	\[
		\tau(2\cdot10)=\tau(20)=6\ne2\cdot4=\tau(2)\cdot\tau(10)
	\]
	В то же время
	\[
		\sigma(2\cdot10)=\sigma(20)=42\ne3\cdot18=\sigma(2)\cdot\sigma(10)
	\]
	Эти вычисления выявляют неприятный факт, что, в общем, не всегда является правдой, что
	\[
		\tau(nm)=\tau(m)\tau(n)
	\]
	и
	\[
		\sigma(nm)=\sigma(m)\sigma(n)
	\]
	Если смотреть с положительной стороны, то равенство всегда выполняется при условии того, что мы придерживаемся взаимно простых $m$ и $n$. Это обстоятельство наводит нас на определение.
	\begin{definition}
		Теоретикочисловая функция $f$ называется мультипликативной, если:
		\[
			f(mn)=f(m)f(n)
		\]
		при  $\gcd(m,n)$=1.
    \end{definition}
	
	В качестве простой иллюстрации мультипликативных функции, достаточно рассмотреть функции $f(n)=1$ и $g(n)=n$ для любых $n\ge1$. Придерживаясь метода индукции, если $f$ мультипликативна и $n_1,n_2,\ldots,n_r$  ~--- положительные целые числа, которые являются попарно взаимнопростыми, то:
	\[
		f(n_1,n_2,\ldots,n_r)=f(n_1)f(n_2)\ldots f(n_r).
	\]
	Мультипликативные функции имеют одно очень весомое преимущество для нас: они полностью вычисляются как только их значения в простых степенях становятся известными. Действительно, если $n>1$ положительная переменная, тогда мы можем записать $n=p_1^{k_1}p_2^{k_2}\ldots p_r^{k_r}$ в канонической форме; начиная с некого $p_i^{k_i}$, они попарно взаимнопростые, мультипликативное свойство гарантирует, что:
	\[
		f(n)=f(p_1^{k_1})f(p_2^{k_2})\ldots f( p_r^{k_r}).
	\]
	Если $f$ мультипликативная функция, которая однозначно не исчезает, то существует такое целое число $n$, что $f(n)\ne0$. Но
	\[
		f(n)=f(n\cdot1)=f(n)f(1).
	\]
	Будучи не нулевым, $f(n)$ может быть отброшен с обеих сторон равенства, что даст $f(1)=1$. Мы хотим обратить внимание на то, что $f(1)=1$ для какой-либо мультипликативной функции определённо не нулевой
	
	Теперь мы установим, что $\tau$ и $\sigma$ обладают мультипликативным свойством.  
	\begin{theorem}
		Функции $\tau$ и $\sigma$ --- обе мультипликативные функции.
	\end{theorem}

	\begin{proof}
		Пусть $m$ и $n$ взаимнопростые целые числа. Поскольку результат тривиально истинен, тогда если или $m$, или $n$ равны $1$, то мы можем предположить, что $m>1$ и $n>1$. Если
		\[
			m=p_1^{k_1}p_2^{k_2}\ldots p_r^{k_r}; n=q_1^{j_1}q_2^{j_2}\ldots q_s^{j_s}
		\]
		являются простыми декомпозициями чисел $m$ и $n$ соответственно, тогда, поскольку $\gcd(m,n)$=1, никакое $p_i$ не может возникнуть среди $q_j$. Придерживаясь тому, что простая декомпозиция произведения $mn$ является 
		\[
			mn=p_1^{k_1}\ldots p_r^{k_r}q_1^{j_1}\ldots q_s^{j_s}
		\]
		Обращаясь к предыдущей теореме, мы находим
		\[
			\tau(mn)=[(k_1+1)\ldots(k_r+1)][(j_1+1)\ldots(j_s+1)]=\tau(m)\tau(n).
		\]
		Аналогичным образом предыдущая теорема даёт нам
		\[
			\sigma(mn)=\left[\frac{p_1^{k_1+1}-1}{p_1-1}\ldots\frac{p_r^{k_r+1}-1}{p_r-1}\right]\left[\frac{q_1^{j_1+1}-1}{q_1-1}\ldots\frac{q_s^{j_s+1}-1}{q_s-1}\right]=\sigma(m)\sigma(n).
		\]
		Таким образом, $\tau$ и $\sigma$ --- мультипликативные функции.
	\end{proof}
		
	Мы продолжим работу, доказывая общий результат мультипликативных функций. Это требует предварительного ввода леммы.
	
	\begin{lemma}
		Если $\gcd(m,n)$=1, тогда набор положительных делителей $mn$ состоит из произведений $d_1d_2$, где $d_1|n,d_2|m$ и $\gcd(d_1,d_2)$=1, кроме того, все эти произведения различны.
	\end{lemma}

	\begin{proof}
		Безошибочно предположить, что $m>1$ и $n>1$. Пусть $m=p_1^{k_1}p_2^{k_2}\ldots p_r^{k_r}$ и  $n=q_1^{j_1}q_2^{j_2}\ldots q_s^{j_s}$ будут их соответствующими простыми декомпозициями. Поскольку простые числа $p_1,\ldots,p_r,q_1,\ldots,q_s$ являются различными, то простой декомпозицией $mn$ является  
		\[
			mn=p_1^{k_1}\ldots p_r^{k_r}q_1^{j_1}\ldots q_s^{j_s}.
		\]
		Следовательно, какой-либо положительный делитель $d$ числа $mn$ будет однозначно представим в форме
		\[
			d=p_1^{a_1}\ldots p_r^{a_r}q_1^{b_1}\ldots q_s^{b_s}, 0\le a_i \le k_i, 0\le b_i\le j_i.
		\]
		Это позволяет нам записать $d$ как $d=d_1d_2$, где $d_1=p_1^{a_1}\ldots p_r^{a_r}$ делит $m$, а  $d_2=q_1^{b_1}\ldots q_s^{b_s}$ делит $n$. Так как никакое $p_i$ не равняется $q_j$, мы, конечно, имеем $\gcd(d_1,d_2)$=1. 
    \end{proof}

	Краеугольным камнем в нашей последующей работе является следующая теорема.
	
	\begin{theorem}
		Если $f$ --- мультипликативная функция, а $F$ определяется, как
		\[
			F(n)=\sum_{d|n}^{}f(d),
		\]
		тогда $F$ также мультипликативна.
	\end{theorem}

	\begin{proof}
		Пусть $m$ и $n$ будут взаимнопростыми положительными целыми числами. Тогда
		\[
			F(mn)=\sum_{d|mn}^{}f(d)=\sum_{d_1|m;d_2|n}^{}f(d_1,d_2),
		\]
		Поскольку каждый делитель $d$ числа $mn$ может быть записан однозначно как произведение делителя $d_1$ числа $m$ и делителя $d_2$ числа $n$, где $\gcd(d_1,d_2)$=1. По определению мультипликативной функции,
		\[
			f(d_1,d_2)=f(d_1)f(d_2).
		\]
		Придерживаясь тому, что
		\[
			F(mn)=\sum_{d_1|m;d_2|n}^{}f(d_1)f(d_2)=\left(\sum_{d_1|m}^{}f(d_1)\right)\left(\sum_{d_2|n}^{}f(d_2)\right)=F(m)F(n).
		\]
	\end{proof}

	Было бы полезно взять перерыв и пробежаться по доказательству этой теоремы с конкретными значениями. Пусть $m=8$, а $n=3$, мы имеем
	\begin{equation} \label{eq1}
	\begin{split}
		F(8\cdot3)=\sum_{d_1|24}^{}f(d)=f(1)+f(2)+f(3)+f(4)+f(6)+f(8)+f(12)+f(24)=\\
		=f(1\cdot1)+f(2\cdot1)+f(1\cdot3)+f(4\cdot1)+f(2\cdot3)+f(8\cdot1)+f(4\cdot3)+f(8\cdot3)=\\ =f(1)f(1)+f(2)f(1)+f(1)f(3)+f(4)f(1)+f(2)f(3)+f(8)f(1)+f(4)f(3)+f(8)f(3)=\\
		=[f(1)+f(2)+f(4)+f8][f(1)+f(3)]=\sum_{d|8}^{}f(d)\cdot\sum_{d|3}^{}f(d)=F(8)F(3).
	\end{split}
	\end{equation}
	Данная теорема даёт обманчиво короткий способ построения доказательства того, что $\tau$ и $\sigma$ мультипликативные функции.
\end{document}